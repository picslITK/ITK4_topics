\documentclass{article}

%%% remove comment delimiter ('%') and specify encoding parameter if required,
%%% see TeX documentation for additional info (cp1252-Western,cp1251-Cyrillic)
%\usepackage[cp1252]{inputenc}

%%% remove comment delimiter ('%') and select language if required
%\usepackage[english,spanish]{babel}

\usepackage{amssymb}
\usepackage{amsmath}
\usepackage[dvips]{graphicx}
%%% remove comment delimiter ('%') and specify parameters if required
%\usepackage[dvips]{graphics}

\begin{document}

%%% remove comment delimiter ('%') and select language if required
%\selectlanguage{spanish}

\noindent $M$: the metric function \newline

\noindent $J$: the moving image \newline

\noindent $I$: the fixed image \newline

\noindent $J^{*}$: the warped moving image \newline

\noindent

\noindent $x$: coordinates in the warped space \newline

\noindent $y = A(x)x+b(x)$ : coordinates in the moving space via a
locally affine transform \newline

\noindent $A(x) = [ a_{ij} ] (x) $ a spatially varying affine
transform. \newline

\noindent $b(x) = b_{i} (x) $ a spatially varying translation
transform. \newline

\noindent

\noindent $ M =  \| I - J(T(x)) \|^2 $ a SSD metric \newline

\noindent$ J^*(x) = J(T(x)) = J(A(x)x+b(x))  $ where $J^*$ is the image
$J$ after the warping is applied. \newline

\noindent We have
\begin{eqnarray}
\frac{1}{2}\frac{\partial} {\partial a_{ij}} \| I - J (  A(x) x + b(x) ) \|^2 = \\ \notag
( I - J (  A(x) x + b(x))~) \frac{d J}{dT} \frac{\partial} {\partial
  a_{ij}}(  A(x) x + b(x) ) = \\ \notag
( I - J (  A(x) x + b(x))~) \frac{d J}{dT} \frac{\partial [a_{ij}]  } {\partial
  a_{ij}} x = \\ \notag
( I - J (  A(x) x + b(x))~) \frac{d J}{dT} \text{\bf{Id}}_{ij} x
\end{eqnarray}
where \text{\bf{Id}}$_{ij}$ represents the $ij$ component of the
identity matrix.   \newline

 With respect to $x_i$ the $i^{th}$ parameter of $x$ :
\begin{eqnarray}
\frac{1}{2}\frac{\partial} {\partial x_{i}} \| I - J (  A(x) x + b(x) ) \|^2 = \\ \notag
( I - J (  A(x) x + b(x))~) \frac{d J}{dT} \frac{\partial} {\partial
  x_{i}}(  A(x) x + b(x) ) = \\ \notag
( I - J (  A(x) x + b(x))~) \frac{d J}{dT} ( \frac{\partial A(x) x } {\partial
  x_{i}} +  \frac{\partial b(x) } {\partial
  x_{i}}    ) = \\ \notag
\text{then apply the product rule}\\ \notag
( I - J (  A(x) x + b(x))~) \frac{d J}{dT} (  A(x) \frac{\partial x } {\partial
  x_{i}}  +  \frac{\partial A(x) } {\partial
  x_{i}}x +  \frac{\partial b(x) } {\partial
  x_{i}}    )  \\ \notag
 \end{eqnarray}

 Assuming $\frac{\partial A(x) } {\partial x_{i}}=0$ and $\frac{\partial b(x) } {\partial x_{i}}=0$, we have

\begin{eqnarray}
\frac{1}{2}\frac{\partial} {\partial x_{i}} \| I - J (  A(x) x + b(x) ) \|^2 = \\ \notag
( I - J (  A(x) x + b(x))~) \frac{d J}{dT} (  A(x) \frac{\partial x } {\partial
  x_{i}}  )  \\ \notag
\end{eqnarray}

\end{document}